\documentclass{uia}

% Todo Notes
\usepackage[colorinlistoftodos,prependcaption,textsize=tiny]{todonotes}

% You can remove this
\usepackage{lipsum}


% Title of your thesis
\title{Deep Reinforcement Learning in Disneyland using TD($\lambda$) derived methods}

% The author(s) of the thesis, Separate authors with \\
\author{Dr. MacStuffins, and Mickey Mouse}

% Supervisors of this thesis
\supervisor{Supervisor 1\\Supervisor 2}

% Which faculty you belong to
\faculty{Faculty of Engineering and Science}

% Which department you belong to
\department{Department of ICT}

% Define which year and month you are delivering your thesis.
\setyear{2018}
\setmonth{January}
\date{\today}

% These are printed at Page 2 and the last page. For most people, these can be left as is.
\copyrightnotice{All rights reserved}
\isbnprinted{}
\isbnelectronic{}
\serialnumber{}

\begin{document}
\pagenumbering{roman}

% Your glossary is defined here. These can be used for keeping track of abbrievations troughout your thesis.
% Example: I then used \gls{ConvNet} for my experiments......
% This will then show up in your glossary.
\newglossaryentry{GAN}{name={GAN},description={Generative Adversarial Networks}}



\maketitle

% Page: Abstract
\begin{abstract}
	\lipsum[1]
	
	\noindent
	\\
	\lipsum[2]
	
	\noindent
	\\
	\lipsum[3]



\end{abstract}
\cleardoublepage

% Page: Preface
\begin{preface}
	This is not mandatory.
	\lipsum[1]
\end{preface}
\cleardoublepage

% Page: Table of Contents
\tableofcontents


% Abbriviations
\printglossaries
\addcontentsline{toc}{chapter}{\numberline{}Glossary}%
\cleardoublepage

% Page: Figures
\listoffigures
\addcontentsline{toc}{chapter}{\numberline{}List of Figures}%
\cleardoublepage
 
% Page: Tables
\listoftables
\addcontentsline{toc}{chapter}{\numberline{}List of Tables}
\cleardoublepage

\addcontentsline{toc}{chapter}{\numberline{}List of Publications}
\listofpublication
\cleardoublepage


% Turn on page counting
\pagenumbering{arabic}

\setlength{\parindent}{0em}
\setlength{\parskip}{1em}

% YOUR THESIS BEGINS HERE!
% YOUR THESIS BEGINS HERE!
% YOUR THESIS BEGINS HERE!
% This layout uses parts to separate the 1. Research, 2. Contributions, and 3. Experiments and Results.
% Discuss with your supervisor if this is needed, as it may be a bit overkill.
% ------------------------------------------------------------------------------------------------

\part{Research Overview}
\chapter{Introduction}
\section{Motivation}
%What is the problem?
%Why is it interesting and important?
%Why is it hard? (E.g., why do naive approaches fail?)
%Why hasn't it been solved before? (Or, what's wrong with previous proposed solutions? How does mine differ?)
%What are the key components of my approach and results? Also include any specific limitations.

\section{Thesis definition}
\label{sec:intro:definition}
Define some goals for your thesis, and maybe some Hypotheses you want to test.
\subsection{Thesis Goals}

\textbf{Goal 1: } ....

\textbf{Goal 2: } ....

\textbf{Goal 3: } ....

\textbf{Goal 4: } ....

\subsection{Hypotheses}

\textbf{Hypothesis 1: } ...

\textbf{Hypothesis 2: } ...

\subsection{Summary}
You may want to summarize your goals and hypotheses here. Can be removed if not needed.

\section{Contributions}
Summarize the contributions proposed by your work 

\section{Thesis outline}
\textit{Outline your thesis structure. Below is an example}

Chapter~\ref{chap:bg} provides preliminary background research for Artificial Neural Networks (\ref{sec:bg:ann}, \ref{sec:bg:convnet}), Generative Models~(\ref{sec:bg:generative}), Markov Decision Process (\ref{sec:bg:mdp}), and Reinforcement Learning (\ref{sec:bg:rl}).

Chapter~\ref{chap:sota} investigates the current state-of-the-art in Deep Neural Networks (\ref{sec:sota:nn}), \gls{RL} (\ref{sec:sota:rl}), \gls{GAN} (\ref{sec:sota:gan}) and Game environments (\ref{sec:sota:gameenv}).

Chapter~\ref{chap:env} outlines the technical specifications for the new game environments Flash RL (\ref{sec:env:flashrl}), Deep Line Wars (\ref{sec:env:deeplinewars}), Deep RTS (\ref{sec:env:deeprts}), and Maze (\ref{sec:env:maze}). In addition, a well established game environment (Section~\ref{sec:env:flappybird}) is introduced to validate experiments conducted in this thesis.

Chapter~\ref{chap:solutions} introduces the proposed solutions for the goals defined in Section~\ref{sec:intro:definition}. Section~\ref{sec:solutions:environments} outlines how the environments are presented as a learning platform. Section~\ref{sec:solutions:capsnet} introduces the proposal to use Capsules in \gls{RL}. Section~\ref{sec:solutions:dqn} describes the Deep Q-Learning algorithm and the implementations used for the experiments in this thesis. Finally, the artificial training data generator is outlined in Section~\ref{sec:solutions:datagen}.


Chapter~\ref{chap:results:ccdn} and \ref{chap:results:dqn} shows experimental results from the work presented in Chapter~\ref{chap:solutions}. 

Chapter~\ref{chap:conclusion} concludes the thesis hypotheses and provides a summary of the work done in this thesis. Section~\ref{sec:conclusion:future_work} outlines the road-map for future research related to the thesis.


\chapter{Background}
\label{chap:bg}
A little introduction of your primary field of research and quickly describe the upcoming sections.
The upcoming sections outlines theory about your research area.

\section{Artificial Neural Networks}
\label{sec:bg:ann}


\subsection{Activation Functions}
\label{sec:bg:ann:activation}

\subsection{Optimization}
\label{sec:bg:ann:optimization}

\subsection{Loss Functions}
\label{sec:bg:ann:loss_functions}

\subsection{Hyper-parameters}
\label{sec:bg:ann:hyper_parameters}


\section{Convolutional Neural Networks}
\label{sec:bg:convnet}

\subsection{Pooling}
\label{sec:bg:convnet:pooling}

\subsection{Summary}
\label{sec:bg:convnet:summary}

\section{Generative Models}
\label{sec:bg:generative_models}


\section{Markov Decision Process}
\label{sec:bg:mdp}

\section{Reinforcement Learning}
\label{sec:bg:rl}


\subsection{Q-Learning}
\label{sec:bg:rl:ql}
\subsection{Deep Q-Learning}
\label{sec:bg:rl:dql}


\chapter{State-of-the-art}
\label{chap:sota}
In this chapter you outline previous work in your field. 

\section{Deep Learning}
\label{sec:sota:nn}

\section{Deep Reinforcement Learning}
\label{sec:sota:rl}


\section{Generative Modeling}
\label{sec:sota:gan}

\section{Capsule Networks}
\label{sec:sota:capsnet}

\section{Game Learning Platforms}
\label{sec:sota:gameenv}

\section{Reinforcement Learning in Games}
\label{sec:sota:games}


\part{Contributions}

\chapter{Environments}
\label{chap:env}
In this case, the thesis propose new games to the Reinforcement Learning community. First explain what your going to present in the Chapter overview, then outline each of your contributions in a section.

\section{FlashRL}
\label{sec:env:flashrl}

\section{Deep Line Wars}
\label{sec:env:deeplinewars}

\section{Deep RTS}
\label{sec:env:deeprts}

\section{Deep Maze}
\label{sec:env:maze}

\section{Flappy Bird}
\label{sec:env:flappybird}


\chapter{Proposed Solutions}
\label{chap:solutions}

Your solution strategy/method

\section{Environments}
\label{sec:solutions:environments}

\section{Capsule Networks}
\label{sec:solutions:capsnet}

\section[Deep Q-Learning]{Deep Q-Learning\footnote{General knowledge of ANN, DQN, and CapsNet from Chapter \ref{chap:bg} is required.}}
\label{sec:solutions:dqn}

\section{Artificial Data Generator}
\label{sec:solutions:datagen}

\part{Experiments and Results}

\chapter[Conditional Convolution Deconvolution Network]{Conditional Convolution Deconvolution Network}
\chaptermark{CCDN}
\label{chap:results:ccdn}

\section{Introduction}
\label{sec:results:datagen:ccdn}

\section{Deep Line Wars}
\label{sec:results:datagen:dlw}

\section{Deep Maze}
\label{sec:results:datagen:ccdn:deepmaze}

\section{FlashRL: Multitask}
\label{sec:results:datagen:ccdn:flashrl}

\section{Flappy Bird}
\label{sec:results:datagen:ccdn:flappybird}

\section{Summary}
Summarize what you just presented

\chapter{Deep Q-Learning}
\label{chap:results:dqn}


\section{Experiments}

\section{Deep Line Wars}

\section{Deep RTS}

\section{Deep Maze}

\section{FlashRL: Multitask}

\section{Flappy Bird}

\section{Summary}

\chapter{Conclusion and Future Work}
\label{chap:conclusion}

\section{Conclusion}
\label{sec:conclusion}

\section{Future Work}
\label{sec:conclusion:future_work}

\cleardoublepage
\renewcommand{\bibname}{References}
\bibliography{library}
\bibliographystyle{plain}
\addcontentsline{toc}{chapter}{\numberline{}References}



\appendix
\chapter*{Appendices}
\addcontentsline{toc}{chapter}{Appendices}
\renewcommand{\thesection}{\Alph{section}}

\section{Hardware Specification}
\label{appendix:hwspec}
	\begin{table}[H]
	\begin{tabular}{|l*{1}{r}|r}
		\hline
		Operating System & Ubuntu 17.10  \\
		\hline
		Processor            & Intel i7-7700K \\
		\hline
		Memory           & 64GB DDR4  \\
		\hline
		Graphics     & 1x NVIDIA GeForce 1080TI   \\
		\hline
	\end{tabular}
\end{table}
%\section{Another appendix Subsection}

% If you have any publications related to your thesis, include them as PDF here
%\part{Publications}
% add them here!
\makebackcover

\end{document}

